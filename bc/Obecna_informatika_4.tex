\clearpage %&latex
\documentclass[a4paper]{article}

\frenchspacing

\usepackage[cp1250]{inputenc}
\usepackage{czech}

\usepackage{a4wide}
\usepackage{amsmath, amsthm, amssymb, amsfonts}
\usepackage[mathcal]{eucal}




\font\bigrm = csr10 scaled \magstep 2
\font\bigbf = csb10 scaled \magstep 4

%Vacsina prostredi je dvojjazicne. V pripade, ze znenie napr pozorovania je pisane po slovensky, malo by byt po slovensky aj oznacenie.

\newenvironment{pozadavky}{\pagebreak[2]\noindent\textbf{Po�adavky}\par\noindent\leftskip 10pt}{\par\bigskip}
\newenvironment{poziadavky}{\pagebreak[2]\noindent\textbf{Po�iadavky}\par\noindent\leftskip 10pt}{\par\bigskip}

\newenvironment{definice}{\pagebreak[2]\noindent\textbf{Definice}\par\noindent\leftskip 10pt}{\par\bigskip}
\newenvironment{definiceN}[1]{\pagebreak[2]\noindent\textbf{Definice~}\emph{(#1)}\par\noindent\leftskip 10pt}{\par\bigskip}
\newenvironment{definicia}{\pagebreak[2]\noindent\textbf{Defin�cia}\par \noindent\leftskip 10pt}{\par\bigskip}
\newenvironment{definiciaN}[1]{\pagebreak[2]\noindent\textbf{Defin�cia~}\emph{(#1)}\par\noindent\leftskip 10pt}{\par\bigskip}

\newenvironment{pozorovani}{\pagebreak[2]\noindent\textbf{Pozorov�n�}\par\noindent\leftskip 10pt}{\par\bigskip}
\newenvironment{pozorovanie}{\pagebreak[2]\noindent\textbf{Pozorovanie}\par\noindent\leftskip 10pt}{\par\bigskip}
\newenvironment{poznamka}{\pagebreak[2]\noindent\textbf{Pozn�mka}\par\noindent\leftskip 10pt}{\par\bigskip}
\newenvironment{poznamkaN}[1]{\pagebreak[2]\noindent\textbf{Pozn�mka~}\emph{(#1)}\par\noindent\leftskip 10pt}{\par\bigskip}
\newenvironment{lemma}{\pagebreak[2]\noindent\textbf{Lemma}\par\noindent\leftskip 10pt}{\par\bigskip}
\newenvironment{lemmaN}[1]{\pagebreak[2]\noindent\textbf{Lemma~}\emph{(#1)}\par\noindent\leftskip 10pt}{\par\bigskip}
\newenvironment{veta}{\pagebreak[2]\noindent\textbf{V�ta}\par\noindent\leftskip 10pt}{\par\bigskip}
\newenvironment{vetaN}[1]{\pagebreak[2]\noindent\textbf{V�ta~}\emph{(#1)}\par\noindent\leftskip 10pt}{\par\bigskip}
\newenvironment{vetaSK}{\pagebreak[2]\noindent\textbf{Veta}\par\noindent\leftskip 10pt}{\par\bigskip}
\newenvironment{vetaSKN}[1]{\pagebreak[2]\noindent\textbf{Veta~}\emph{(#1)}\par\noindent\leftskip 10pt}{\par\bigskip}

\newenvironment{dusledek}{\pagebreak[2]\noindent\textbf{D�sledek}\par\noindent\leftskip 10pt}{\par\bigskip}
\newenvironment{dosledok}{\pagebreak[2]\noindent\textbf{D�sledok}\par\noindent\leftskip 10pt}{\par\bigskip}

\newenvironment{dokaz}{\pagebreak[2]\noindent\leftskip 10pt\textbf{D�kaz}\par\noindent\leftskip 10pt}{\par\bigskip}
\newenvironment{dukaz}{\pagebreak[2]\noindent\leftskip 10pt\textbf{D�kaz}\par\noindent\leftskip 10pt}{\par\bigskip}

\newenvironment{priklad}{\pagebreak[2]\noindent\textbf{P��klad}\par\noindent\leftskip 10pt}{\par\bigskip}
\newenvironment{prikladSK}{\pagebreak[2]\noindent\textbf{Pr�klad}\par\noindent\leftskip 10pt}{\par\bigskip}
\newenvironment{priklady}{\pagebreak[2]\noindent\textbf{P��klady}\par\noindent\leftskip 10pt}{\par\bigskip}
\newenvironment{prikladySK}{\pagebreak[2]\noindent\textbf{Pr�klady}\par\noindent\leftskip 10pt}{\par\bigskip}

\newenvironment{algoritmusN}[1]{\pagebreak[2]\noindent\textbf{Algoritmus~}\emph{(#1)}\par\noindent\leftskip 10pt}{\par\bigskip}
%obecne prostredie, ktore ma vyuzitie pri specialnych odstavcoch ako (uloha, algoritmus...) aby nevzniklo dalsich x prostredi
\newenvironment{obecne}[1]{\pagebreak[2]\noindent\textbf{#1}\par\noindent\leftskip 10pt}{\par\bigskip}


\newenvironment{penumerate}{
\begin{enumerate}
  \setlength{\itemsep}{1pt}
  \setlength{\parskip}{0pt}
  \setlength{\parsep}{0pt}
  %\setlength{\topsep}{200pt}
  \setlength{\partopsep}{200pt}
}{\end{enumerate}}

\def\pismenka{\numberedlistdepth=2} %pouzit, ked clovek chce opismenkovany zoznam...

\newenvironment{pitemize}{
\begin{itemize}
  \setlength{\itemsep}{1pt}
  \setlength{\parskip}{0pt}
  \setlength{\parsep}{0pt}
}{\end{itemize}}

\definecolor{gris}{gray}{0.95}
\newcommand{\ramcek}[2]{\begin{center}\fcolorbox{white}{gris}{\parbox{#1}{#2}}\end{center}\par}
 \clearpage
\title{\LARGE U�ebn� texty k st�tn� bakal��sk� zkou�ce \\ Obecn� informatika \\ Datab�ze}
\begin{document}
\maketitle
\newpage
\setcounter{section}{3}
\section{Datab�ze}
\begin{pozadavky}
\begin{pitemize}
\item Podstata a architektury DB syst�m�.
\item Konceptu�ln�, logick� a fyzick� �rove� pohled� na data.
\item Rela�n� datov� model, rela�n� algebra.
\item Algoritmy n�vrhu sch�mat relac�, norm�ln� formy, referen�n� integrita.
\item Z�klady SQL.
\item Transak�n� zpracov�n�, vlastnosti transakc�.
\item Organizace dat na vn�j�� pam�ti, B-stromy a jejich varianty.
\end{pitemize}
\end{pozadavky}
\subsection{Podstata a architektury DB system�}

Zdroje: Wikipedie, slidy Dr. T. Skopala k Datab�zov�m syst�m�m
\bigskip

\begin{e}{Definice}{0}{Datab�ze}
Datab�ze je logicky uspo��dan� (integrovan�) kolekce navz�jem souvisej�c�ch dat. Je sebevysv�tluj�c�, proto�e data jsou uchov�v�na spole�n� s�popisy, zn�m�mi jako metadata (tak� sch�ma datab�ze). Data jsou ukl�d�na tak, aby na nich bylo mo�n� prov�d�t strojov� dotazy -- z�skat pro n�jak� parametry vyhovuj�c� podmno�inu z�znam�.

N�kdy se slovem \uv{datab�ze} mysl� obecn� cel� datab�zov� syst�m.
\end{e}

\begin{e}{Definice}{0}{Syst�m ��zen� b�ze dat}
Syst�m ��zen� b�ze dat (S�BD, anglicky database management system, DBMS) je obecn� softwarov� syst�m, kter� ��d� sd�len� p��stup k�datab�zi, a poskytuje mechanismy, pom�haj�c� zajistit bezpe�nost a integritu ulo�en�ch dat. Spravuje datab�zi a zaji��uje prov�d�n� dotaz�.
\end{e}

\begin{e}{Definice}{0}{Datab�zov� syst�m}
Datab�zov�m syst�mem rozum�me trojici, sest�vaj�c� z:
\begin{pitemize}
    \item datab�ze
    \item syst�mu ��zen� b�ze dat
    \item chud�ka admina
\end{pitemize}
\end{e}

\begin{obecne}{Smysl datab�z�}
Hlavn�m smyslem datab�ze je schra�ovat datov� z�znamy a informace za ��elem:
\begin{pitemize}
    \item sd�len� dat v�ce u�ivateli,
    \item zaji�t�n� unifikovan�ho rozhran� a jazyk� definice dat a manipulace s daty,
    \item znovuvyu�itelnosti dat,
    \item bezespornosti dat a
    \item sn�en� objemu dat (odstran�n� redundance).
\end{pitemize}
\end{obecne}

\subsubsection*{Datab�zov� modely}

\begin{e}{Definice}{0}{sch�ma, model}
Typicky pro ka�dou datab�zi existuje struktur�ln� popis druh� dat v n� udr�ovan�ch, ten naz�v�me \emph{sch�ma}. Sch�ma popisuje objekty reprezentovan� v datab�zi a vztahy mezi nimi. Je n�kolik mo�n�ch zp�sob� organizace sch�mat (modelov�n� datab�zov� struktury), zn�m�ch jako \emph{modely}. V modelu jde nejen o zp�sob strukturov�n� dat, definuje se tak� sada operac� nad daty provediteln�. Rela�n� model nap��klad definuje operace jako \uv{select} nebo \uv{join}. I kdy� tyto operace se nemusej� p��mo vyskytovat v dotazovac�m jazyce, tvo�� z�klad, na kter�m je jazyk postaven. Nejd�le�it�j�� modely v t�to sekci pop�eme.
\end{e}

\begin{e}{Pozn�mka}{0}{0}
V�t�ina datab�zov�ch syst�m� je zalo�ena na jednom konkr�tn�m modelu, ale ��m d�l �ast�j�� je podpora v�ce p��stup�. Pro ka�d� logick� model existuje v�ce fyzick�ch p��stup� implementace a v�t�ina syst�m� dovol� u�ivateli n�jakou �rove� jejich kontroly a �prav, proto�e toto m� velk� vliv na v�kon syst�mu. P��kladem nech� jsou indexy, provozovan� nad rela�n�m modelem.
\end{e}

\begin{obecne}{\uv{Ploch�} model}
Toto sice nevyhovuje �pln� definici modelu, p�esto se jako trivi�ln� p��pad uv�d�. P�edstavuje jedinou dvoudimension�ln� tabulku, kde data v jednom sloupci jsou pova�ov�na za popis stejn� vlastnosti (tak�e maj� podobn� hodnoty) a data v jednom ��dku se uva�uj� jako popis jedin�ho objektu.
\end{obecne}

\begin{obecne}{Rela�n� model}
Rela�n� model je zalo�en na predik�tov� logice a teorii mno�in. V�t�ina fyzicky implementovan�ch datab�zov�ch syst�m� ve skute�nosti pou��v� jen aproximaci matematicky definovan�ho rela�n�ho modelu. Jeho z�kladem jsou \emph{relace} (dvoudimension�ln� tabulky), \emph{atributy} (jejich pojmenovan� sloupce) a \emph{dom�ny} (mno�iny hodnot, kter� se ve sloupc�ch m��ou objevit). Hlavn� datovou strukturou je tabulka, kde se nach�z� informace o n�jak� konkr�tn�  t��d� entit. Ka�d� entita t� t��dy je potom reprezentov�na ��dkem v tabulce -- $n$-tic� atribut�.

V�echny relace (tj. tabulky) mus� spl�ovat z�kladn� pravidla -- po�ad� sloupc� nesm� hr�t roli, v tabulce se nesm� vyskytovat identick� ��dky a ka�d� ��dek mus� obsahovat jen jednu hodnotu pro ka�d� sv�j atribut. Rela�n� datab�ze obsahuje v�ce tabulek, mezi kter�mi lze popisovat vztahy (v�ech r�zn�ch kardinalit, tj. $1:1$, $1:n$ apod.). Vztahy vznikaj� i implicitn� nap�. ulo�en�m stejn� hodnoty jednoho atributu do dvou ��dk� v tabulce. K tabulk�m lze p�idat informaci o tom, kter� podmno�ina atribut� funguje jako \emph{kl��}, tj. unik�tn� identifikuje ka�d� ��dek, n�kter� z kl��� m��e b�t ozna�en jako prim�rn�. N�kter� kl��e m��ou m�t n�jak� vztah k vn�j��mu sv�tu, jin� jsou jen pro vnit�n� pot�eby sch�matu datab�ze (generovan� ID).
\end{obecne}

\begin{obecne}{Hierarchick� model}
V hierarchick�m modelu jsou data organizov�na do stromov� struktury -- ka�d� uzel m� odkaz na nad��zen� (k popisu hierarchie) a set��d�n� pole z�znam� na stejn� �rovni. Tyto struktury byly pou��v�ny ve star�ch mainframeov�ch datab�z�ch, nyn� je m��eme vid�t nap� ve struktu�e XML dokument�. Dovoluj� vztahy $1:N$ mezi dv�ma druhy dat, co� je velice efektivn� k popisu r�zn�ch re�ln�ch vztah� (obsahy, �azen� odstavc� textu, t��d�n� informace). Nev�hodou je ale nutnost zn�t celou cestu k z�znamu ve struktu�e a neschopnost syst�mu reprezentovat redundance v datech (strom nem� cykly).
\end{obecne}

\begin{obecne}{S�ov� model}
S�ov� model organizuje data pomoc� dvou hlavn�ch prvk�, \emph{z�znam�} a \emph{mno�in}. Z�znamy obsahuj� pole dat, mno�iny definuj� vztahy $1:N$ mezi z�znamy (jeden \emph{vlastn�k}, mnoho \emph{prvk�}). Z�znam m��e b�t vlastn�kem i prvkem v n�kolika r�zn�ch mno�in�ch. Jde vlastn� o variantu hierarchick�ho modelu, proto�e s�ov� model je tak� zalo�en na konceptu v�ce struktur ni��� �rovn� z�visl�ch na struktur�ch �rovn� vy���. U� ale umo��uje reprezentovat i redundantn� data. Operace nad t�mto modelem prob�haj� \uv{naviga�n�m} stylem: program si uchov�v� svoji sou�asnou pozici mezi z�znamy a postupuje podle z�vislost�, ve kter�ch se dan� z�znam n�ch�z�. Z�znamy mohou b�t i vyhled�v�ny podle kl��e. 

Fyzicky jsou v�t�inou mno�iny -- vztahy -- reprezentov�ny p��mo ukazateli na um�st�n� dat na disku, co� zaji��uje vysok� v�kon p�i vyhled�v�n�, ale zvy�uje n�klady na reorganizace. Smysl s�ov� navigace mezi objekty se pou��v� i v objektov�ch modelech.
\end{obecne}

\begin{obecne}{Objektov� model}
Objektov� model je aplikac� p��stup� zn�m�ch z objektov�-orientovan�ho programov�n�. Je zalo�en na sbli�ov�n� programov� aplikace a datab�ze, hlavn� ve smyslu pou�it� datov�ch typ� (objekt�) definovan�ch na jednom m�st�; ty zp��stup�uje k pou�it� v n�jak�m b�n�m programovac�m jazyce. Odstran� se tak nutnost zbyte�n�ch konverz� dat. P�in�� do datab�z� tak� v�ci jako zapouzd�en� nebo polymorfismus. Probl�mem objektov�ch model� je neexistence standard� (nebo sp� produkt�, kter� by je implementovaly).

Kombinac� objektov�ho a rela�n�ho p��stupu vznikaj� \emph{objektov�-rela�n�} datab�ze -- rela�n� datab�ze, dovoluj�c� u�ivateli definovat vlastn� datov� typy a operace na nich. Obsahuj� pak hybrid mezi procedur�ln�m a dotazovac�m programovac�m jazykem.

\end{obecne}


\subsubsection*{Architektury datab�zov�ch syst�m�}

Zdroj: Wiki �VUT (st�tnice na FELu ;-))
\bigskip

Architektury datab�zov�ch syst�m� se obecn� d�l� na 
\begin{pitemize}
    \item \emph{centralizovan�} (kde se datab�ze p�edpokl�d� fyzicky na jednom po��ta�i) a
    \item \emph{distribuovan�},
\end{pitemize}
p��padn� na
\begin{pitemize}
    \item \emph{jednou�ivatelsk�} a
    \item \emph{v�ceu�ivatelsk�}.
\end{pitemize}

\begin{obecne}{Distribuovan� datab�zov� syst�my}
\emph{Distribuovan� syst�m ��zen� b�ze dat} je vlastn� speci�ln�m p��padem obecn�ho distribuovan�ho v�po�etn�ho syst�mu. Jeho implementace zahrnuje fyzick� rozlo�en� dat (v�etn� mo�n�ch replikac� datab�ze) na v�ce po��ta�� -- \emph{uzl�}, p�i�em� jejich popis je integrov�n v glob�ln�m datab�zov�m sch�matu. Data v uzlech mohou b�t zpracov�v�na lok�ln�mi S�BD, komunikace je organizov�na v s�ov�m provozu pomoc� speci�ln�ho softwaru, kter� um� zach�zet s distribuovan�mi daty. Fyzicky se �e�� rozlo�en� do uzl�, sv�zan�ch komunika�n�mi kan�ly, a jeho transparence (neviditelnost -- navenek se m� tv��it jako jednolit� syst�m). Ka�d� uzel v s�ti je s�m o sob� datab�zov� syst�m a z ka�d�ho uzlu lze zp��stupnit data kdekoliv v s�ti.

D�le se d�l� na dva typy:
\begin{pitemize}
    \item Federativn� datab�ze -- neexistuje glob�ln� sch�ma ani centr�ln� ��d�c� autorita, ��zen� je tak� distribuovan�.
    \item Heterogenn� datab�zov� syst�my -- jednotliv� autonomn� S�BD existuj� (vznikly nez�visle na sob�) a jsou integrov�ny, aby spolu mohly komunikovat.
\end{pitemize}

V�hodou oproti centralizovan�m syst�m�m je vy��� efektivita (data mohou b�t ulo�ena bl�zko m�sta nej�ast�j��ho pou��v�n�), zv��en� dostupnost, v�konnost a roz�i�itelnost; nev�hodou z�st�v� probl�m slo�itosti implementace, distribuce ��zen� a ni��� bezpe�nost takov�ch �e�en�.
\end{obecne}

\begin{obecne}{V�ceu�ivatelsk� datab�zov� syst�my}
\emph{V�ceu�ivatelsk�} jsou takov� syst�my, kter� umo��uj� v�cen�sobn� u�ivatelsk� p��stup k dat�m ve stejn�m okam�iku. V d�sledku mo�n�ho sou�asn�ho p��stupu v�ce u�ivatel� je nutn� syst�m zabezpe�it tak, aby i nad�le zaji��oval integritu a konzistenci ulo�en�ch dat. Existuj� obecn� dva mo�n� p��stupy:
\begin{pitemize}
    \item Uzamyk�n� -- D��ve �asto pou��van� metoda zalo�en� na uzamyk�n� aktualizovan�ch z�znam�, v p��pad� masivn�ho vyu�it� aktualiza�n�ch p��kaz� u n� ale m��e doch�zet k zna�n�m prodlev�m. 
    \item Multiversion Concurency Control -- Modern�j�� vyn�lez. Jeho princip spo��v� v tom, �e p�i po�adavku o aktualizaci z�znamu v tabulce je vytvo�ena kopie z�znamu, kter� nen� pro ostatn� u�ivatele a� do proveden�ho commitu viditeln�.
\end{pitemize}
\end{obecne}

\subsection{Konceptualn�, logick� a fyzick� �rove� pohledu na data}

TODO: sjednotit terminologii, snad to popisuje to co tu m� b�t, ale zdroje jsou pochybn� (Wikipedie tady neodv�d� zrovna ide�ln� pr�ci a �VUT Wiki se moc nerozepisuje).

\begin{e}{Definice}{0}{Datov� modelov�n�}
\emph{Datov� modelov�n�} je proces vytvo�en� konkr�tn�ho datov�ho modelu (sch�matu) datab�ze pomoc� aplikace n�jak�ho abstraktn�ho datab�zov�ho modelu. Datov� modelov�n� zahrnuje krom� definice struktury a organizace dat je�t� dal�� implictin� nebo explicitn� omezen� na data do struktury ukl�dan�. 
\end{e}

\begin{obecne}{Vrstvy modelov�n�}
Druhy datov�ch model� mohou b�t t�� typ�, podle t�� r�zn�ch pohled� na datab�ze (t�i \uv{vrstvy}, kter� se navz�jem dopl�uj�):
\begin{pitemize}
    \item konceptu�ln� sch�ma (datov� model) -- nejabstraktn�j��, popisuje v�znam organizace datab�ze -- t��dy entit a jejich vztahy.
    \item logick� sch�ma -- popisuje v�znam konceptu�ln�ho sch�matu z hlediska datab�zov� implementace -- popisy tabulek, programov�ch t��d nebo XML tag� (podle zvolen�ho datab�zov�ho modelu)
    \item fyzick� sch�ma -- nejkonkr�tn�j��, popisuje fyzick� ulo�en� dat a stroje na kter�ch syst�m pob��.
\end{pitemize}
Na tomto rozd�len� je d�le�it� nez�vislost jednotliv�ch vrstev -- tak�e se implementace jedn� z nich m��e zm�nit, ani� by bylo nutn� v�razn� upravovat ostatn� (samoz�ejm� mus� z�stat konzistetn� vzhledem k ostatn�m vrstv�m). B�hem implementace n�jak� datab�zov� aplikace se za��n� vytvo�en�m konceptu�ln�ho sch�matu, pokra�uje jeho up�esn�n� logick�m sch�matem a naknec jeho fyzickou implementac� podle fyzick�ho sch�matu (modelu).
\end{obecne}

\begin{e}{Pozn�mka}{0}{0}
V tomto pohledu (kter� je podle standardu ANSI z r. 1975) jsou datab�zov� modely, popsan� v p�edchoz� sekci, p��klady abstraktn�ch logick�ch datov�ch model�. N�kde je v�ak tato �rove� ozna�ov�na jako \uv{fyzick�} a \uv{jin� logick�} se vt�sn� je�t� mezi ni a konceptu�ln�.
\end{e}


\begin{obecne}{Konceptu�ln� sch�ma}
\emph{Konceptu�ln� sch�ma} (datov� model) popisuje podstatn� objekty (\emph{t��dy entit}, \uv{koncepty}), jejich charakteristiky (\emph{atributy}) a vztahy mezi nimi (asociace mezi dvojicemi t��d entit). Nepopisuje p��mo implementaci v datab�zi, jen v�znam n�jak�ho celku, kter� bude datab�z� p�edstavov�n. Jde o modelov�n� \uv{datov� reality}, z pohledu u�ivatele (analytika, konstrukt�ra datab�ze).

\medskip
\begin{e}{P��klady}{0}{0}
P�r p��klad� vztah� mezi t��dami entit (z Wikipedie):
\begin{pitemize}
    \item Each PERSON may be the vendor in one or more ORDERS.
    \item Each ORDER must be from one and only one PERSON.
    \item PERSON is a sub-type of PARTY. (Meaning that every instance of PERSON is also an instance of PARTY.)
\end{pitemize}
\end{e}
De-facto standardem pro konceptu�ln� datov� modelov�n� jsou \emph{ER-diagramy} (entity-relationship diagramy). Hod� se hlavn� pro \uv{ploch�} form�tovan� data (tak�e t�eba pro objektov� nebo rela�n� datab�ze, ale ne pro XML apod.). Pou��vaj� dva typy \uv{objekt�} -- \emph{entity} (t��dy entit) a \emph{vztahy}. Jde o obdobu UML z objektov�ho programov�n�. P��klad ER-diagramu se vztahem dvou entit je na n�sleduj�c�m obr�zku (popisuje i dal�� vlastnosti -- atributy entit a kardinality vztah�):
\begin{center}
\includegraphics[width=14cm]{informatika/databazy/obrazky/er-schema.png}

(Obr�zek je upraven�, roz���en� a popsan� p��klad ze slid� Dr. T. Skopala k Datab�zov�m syst�m�m)
\end{center}
\end{obecne}

\begin{obecne}{Logick� sch�ma}
\emph{Logick� sch�ma} je datov� model organizace n�jak�ho specifick�ho celku pomoc� jednoho z datab�zov�ch model� -- podle datab�zov�ch model� popsan�ch v p�edchoz� sekci, tj. nap�. pomoc� rela�n�ch tabulek, objektov�ch t��d nebo XML. Svoj� �rovn� abstrakce se nach�z� mezi konceptu�ln�m a fyzick�m sch�matem.
\end{obecne}

\begin{obecne}{Fyzick� sch�ma}
\emph{Fyzick� datov� modely} jsou modely, ktere pou��vaj� databazov� stroje sm�rem k ni���m vrstv�m (opera�n�ho) syst�mu. V z�sad� jde o r�zn� zp�soby fyzick�ho ulo�en� dat (tedy sch�mata organizace soubor�) -- sekven�n� soubory, B-stromy apod.
\end{obecne}


\subsection{Rela�n� datov� model, rela�n� algebra}


\begin{prikladN}{(IOI 10.2.2011)Relacni datovy model, relacni agebra
a)Popi�te relacn� datov� model. Co je relacn� algebra a k cemu slouzi?
b) Jak� zn�te operace RA? Formulujte predpoklady pouzitelnosti kazd� operace. Definujte
vysledek dan� operace nad relacemi.
c) Kter� operace jsou nezbytn� pro zachovan� vyjadrovaci sily jazyka - respektive, je mozn�
nekter� operace formulovat pomoci ostatn�ch operaci? Kter� a jak?}
\\
  
\end{prikladN}


\begin{prikladN}{(IOI 21.6.2011) Co je to rela�n� algebra a jak� operace pou��v�?
5.2 U ka�d� operace popi�te sch�ma relace, na kter� se d� tato operace pou��t a definujte v�sledek operace.
5.3 Jsou v�echny operace nezbytn� pro zachov�n� vyjad�ovac� s�ly jazyka? (Pokud ne, kter� jsou?)
5.4 �emu odpov�d� operace p�irozen� spojen� na relac�ch, kter� maj� toto�n� sch�ma?
5.5 �emu odpov�d� operace p�irozen� spojen� na relac�ch, jejich� sch�mata jsou disjunktn�?}
\\
  
\end{prikladN}

\subsection{Algoritmy n�vrhu sch�mat relac�}


\subsubsection*{Norm�ln� formy}

\begin{obecne}{Normalizace, anom�lie}
Normalizace datab�z� je technika n�vrhu rela�n�ch datab�zov�ch tabulek, pri kter� se minimalizuj� duplicity informac� - a zamezuje se tak nekonzistentnosti dat. Stupn� normalizace se \uv{popisuj�} pomoc� \emph{norm�ln�ch forem} - ��m vy��� forma, t�m vy��� striktnost...

Probl�my �e�en� normalizac�:
\begin{pitemize}
	\item \emph{update anomaly} -- pokud se zm�n� jedna kopie redundantn�ch dat, je t�eba
zm�nit i ostatn� kopie, jinak se datab�ze stane nekonzistentn�, p�.: tabulka (�lov�k, adresa, skill); kdyby se nevykonal update spr�vn�, m��e tabulka z�stat v nekonzistentn�m stavu (nap�. by se mohly zm�nit jen n�kter� adresy jednoho �lov�ka)
	\item \emph{insertion anomaly} -- p�i vlo�en� dat p��slu�ej�c�ch jedn� entit� je pot�eba z�rove� vlo�it data i o jin� entit�, nap�. v tabulce (fakulta, datum zalo�en�, kurz) m��eme zaznamenat jen data pro fakulty, kter� maj� kurzy...
	\item \emph{deletion anomaly} -- P�i vymaz�n� dat p��slu�ej�c�ch jedn� entit� je pot�eba vymazat
data pat��c� jin� entit�. V p�edchoz� tabulce bude fakulta vymaz�na �pln�, kdy� se v�emi kurzy.
\end{pitemize}

Ide�ln� by rela�n� datab�ze m�la b�t navr�ena tak, aby vylu�ovala mo�nost takov�ch anomali�. Normalizace obvykle zahr�uje dekomponov�n� nenormalizovan� tabulky na dv� nebo v�ce tabulek takov�ch, �e po jejich spojen� (join) dostaneme v�echny p�vodn� informace.

Abychom mohli definovat norm�ln� formy, pot�ebujeme zn�t funk�n� z�vislosti jednotliv�ch atribut� entit rela�n� datab�ze a v�d�t, kter� atributy jsou kl��ov� a kter� ne.
\end{obecne}

\begin{definiceN}{Funk�n� z�vislosti}

\medskip\noindent
�ekneme, �e atribut \emph{B} je \textbf{funk�n� z�visl�} na atributu \emph{A}
(zna��me $A\rightarrow B$), jestli�e pro ka�dou hodnotu atributu \emph{A}
existuje pr�v� jedna hodnota atributu \emph{B}. Roz���en� funk�n� z�vislosti se definuj� pro mno�inu atribut� (pro ka�dou $n$-tici atribut� z n�jak� mno�iny existuje pr�v� jedna hodnota z�visl�ho(z�visl�ch) atributu(atribut�)).

Funk�n� z�vislosti spl�uj� tzv. \emph{Armstrongova pravidla}, co� zahrnuje pro mno�iny atribut� $X,Y,Z$:
\begin{penumerate}
    \item trivi�ln� z�vislost: $X\supseteq Y\ \Rightarrow\ X\to Y$
    \item transitivitu: $X\to Y \wedge Y\to Z\ \Rightarrow\ X\to Z$
    \item kompozici: $X\to Y \wedge X\to Z\ \Rightarrow X\to YZ$
    \item dekompozici: $X\to YZ \ \Rightarrow \ X\to Y \wedge X\to Z$
\end{penumerate}
\end{definiceN}

\begin{definiceN}{Kl��}
\textbf{Nadkl��em}, n�kdy t� \textbf{superkl��em}, sch�matu $A$ rozum�me ka�dou
podmno�inu mno�iny $A$, na n� $A$ funk�n� z�vis�. Jinak �e�eno nadkl�� je mno�ina
atribut�, kter� jednozna�n� ur�uje ��dek tabulky.

\textbf{Kl��}, nebo tak� \textbf{potenci�ln� kl��}(candidate key), sch�matu $A$
je takov� nadkl�� sch�matu $A$, jeho� ��dn� vlastn� podmno�ina nen� nadkl��em
$A$. �ili minim�ln� nadkl��.

Ka�d� atribut, kter� je obsa�en alespo� v jednom potenci�ln�m kl��i se naz�v�
\textbf{kl��ov�}, ostatn� atributy jsou \textbf{nekl��ov�}.
\end{definiceN}

\begin{definiceN}{Norm�ln� formy}
\begin{pitemize}
	\item \emph{Prvn� norm�ln� forma} \\ -- Tabulka je v prvn� norm�ln� form�, jestli�e lze do ka�d�ho pole dosadit pouze jednoduch� datov� typ (jsou d�le ned�liteln�). To zahrnuje i neexistenci v�ce sloupc� tabulky se stejn�m druhem obsahu:
$$
\left.\begin{aligned}
\textrm{(manager, pod��zen�1, pod��zen�2, pod��zen�3)} \\ 
\textrm{(manager, pod��zen�-vice\_hodnot\_v\_jednom\_sloupci)} \\
\end{aligned}\right\} \rightarrow \textrm{(manager, pod��zen�)}
$$
	\item \emph{Druh� norm�ln� forma} \\ 
-- Existuje kl�� a v�echna nekl��ov� pole jsou funkc� cel�ho kl��e (a tedy ne jen jeho ��st�). 
$$\textrm{(custID, name, address, city, state, zip)} \rightarrow
\begin{aligned}&\textrm{(custID, name, address, zip)}\\
&+ \textrm{(zip, city, state)}
\end{aligned}$$
	\item \emph{T�et� norm�ln� forma} \\ -- Tabulka je ve t�et� norm�ln� form�, jestli�e ka�d� nekl��ov� atribut nen� transitivn� z�visl� na ��dn�m kl��i sch�matu (resp. ka�d� nekl��ov� atribut je p��mo z�visl� na kl��i sch�matu) neboli je-li ve druh� norm�ln� form� a z�rove� neexistuje jedin� z�vislost nekl��ov�ch sloupc� tabulky. 
$$\textrm{(deptID, deptName, managerID, hireDate)} \rightarrow \textrm{(deptID, deptName, managerID)}$$
Atribut \uv{hireDate} je sice funk�n� z�visl� na kl��i deptID, ale jen proto, �e hireDate z�vis� na managerID, kter� z�vis� na deptID.
	\item \emph{Boyce-Coddova norm�ln� forma}\\ -- Pro ka�dou netrivi�ln� z�vislost $X \rightarrow Y$ plat�, �e $X$ obsahuje kl�� sch�matu $R$ ($X$ je nadkl��).
\end{pitemize}
\end{definiceN}

\subsubsection*{Algoritmy n�vrhu sch�mat relac�}

Sch�mata relac� by m�la b�t navrhov�na tak, aby odpov�dala p�edem p�ipraven�mu konceptu�ln�mu modelu (nap�. pomoc� ER diagram�) a z�rove� pokud mo�no spl�ovala co nejp��sn�j�� po�adavky na norm�ln� formy. Pro modelov�n� rela�n� datab�ze existuj� dva p��stupy:
\begin{penumerate}
    \item Z�sk�n� mno�iny rela�n�ch sch�mat (ru�n� nebo p�evodem z nap�. ER diagramu) a prov�d�n� normalizace pro ka�dou tabulku zvl᚝
    \item N�vrh tzv. univerz�ln�ho sch�matu datab�ze -- jedna velk� tabulka pro celou datab�zi (v�. platn�ch funk�n�ch z�vislost�) a normalizace prov�d�n� glob�ln�
\end{penumerate}
Prvn� mo�nost je relativn� intuitivn� (s ER diagramy) a jednoduch�, ale hroz� riziko p��li�n�ho rozdroben� datab�ze na velk� po�et mal�ch tabulek (a nadbyte�n� i vzhledem k po�adovan� norm�ln� form�). V druh�m zp�sobu jsou entity jednotliv�ch relac� \uv{vypozorov�ny} jako efekt funk�n�ch z�vislost�, co� nen� p��li� pr�hledn� a jednodu�e provediteln�, ale minimalizuje to �anci na rozdroben� datab�ze. Oba p��stupy lze tak� zkombinovat -- p�ev�st ER model datab�ze do sch�mat a n�kter� (nebo a� v�echna) potom p�ed normalizac� slou�it.

\begin{obecne}{Normalizace}
Jedin�m zp�sobem, jak u n�jak�ho obecn�ho rela�n�ho sch�matu dos�hnout norm�ln� formy (obecn� se po�aduje v�t�inou 3NF nebo BCNF), je rozd�len� na n�kolik podsch�mat. D� se to prov�st ru�n� nebo algoritmicky a existuje v�ce p��stup� podle po�adavku na norm�ln� formu, \emph{bezztr�tovost} (dekompozice relace $R( A, F )$ do $R_1(A_1,F_1)$ a $R_2(A_2,F_2)$ je bezeztr�tov�, kdy� $A1 \cap A2\to A1$ nebo $A1 \cap A2 \to A2$, tedy op�tovn�m spojen�m do p�vodn� relace nevzniknou dal�� ��dky) nebo \emph{pokryt� z�vislost�} (dekompozice $R(A,F)$ do $R_1(A_1,F_1)$ zachov�v� pokryt� z�vislost�, kdy� $F^{+}=F^{+}_1\cup F^{+}_2$ -- nesm� se ztratit z�vislost ani v r�mci d�l��ho sch�matu, ani jdouc� nap��� sch�maty).
\end{obecne}

\begin{algoritmusN}{Dekompozice}
Dekompozice je algoritmus, kter� rela�n� sch�ma p�evede do Boyce-Coddovy norm�ln� formy. Zaru�uje zachov�n� bezeztr�tovosti, ale u� ne pokryt� z�vislost� (bez ohledu na algoritmus toto u BCNF n�kdy nen� mo�n�). Jeho b�h vypad� n�sledovn�:
\begin{penumerate}
    \item Vyber n�jak� sch�ma, kter� nen� v BCNF.
    \item Vezmi pro n�j nekl��ovou z�vislost $X\to Y$ (tak �e $X$ nen� kl��) a dekomponuj podle n� -- vyho� ze sch�matu $Y$ a dej $XY$ do zvl�tn� tabulky.
    \item Opakuj od kroku 1, dokud existuje sch�ma, kter� nen� v BCNF.
\end{penumerate}
\end{algoritmusN}

\begin{algoritmusN}{Synt�za}
Algoritmus synt�zy obecn� dosahuje t�et� norm�ln� formy a zachov�v� pokryt� z�vislost� (ale ne bezeztr�tovost). Pro rela�n� sch�ma $R$ s mno�inou funk�n�ch z�vislost� $F$ vypad� n�sledovn�:
\begin{penumerate}
    \item Ud�lej minim�ln� pokryt� $F$ (vzhledem k tranzitivit�), nazvi ho $G$.
    \item Slu� funk�n� z�vislosti z $G$ se stejnou levou stranou a z ka�d� vytvo� jedno sch�ma. 
    \item Zaho� sch�mata, kter� jsou podmno�iny jin�ch.
\end{penumerate}
Nakonec je mo�n� slou�it sch�mata s funk�n� ekviv. kl��i ($K1 \leftrightarrow K2$), ale m��e to poru�it norm�ln� formu, kter� bylo dosa�eno! Pro zachov�n� bezeztr�tovosti lze do p�idat n�jak� sch�ma, obsahuj�c� univerz�ln� kl�� cel�ho p�vodn�ho (ned�len�ho) sch�matu.
\end{algoritmusN}

\begin{poznamka}
Pro nalezen� minim�ln�ho pokryt� atribut� se pou��v� pomocn� algoritmus, kter� se chov� takto:
\begin{penumerate} 
    \item Dekomponuj v�echny funk�n� z�vislosti na element�rn� (na prav� stran� je jen jeden sloupec)
    \item Odstra� z nich redundantn� atributy (takov� z lev� strany, kter� funk�n� z�vis� na jin�ch z lev� strany)
    \item Odstra� redundantn� funk�n� z�vislosti (tj. takov�, kter� jsou tranzitivn�m d�sledkem jin�ch -- prav� strana funk�n� z�vis� na lev�, i kdy� z mno�iny funk�n�ch z�vislost� onu redundantn� odstran�m)
\end{penumerate}
Pro druh� i t�et� krok je pot�eba z�skat \emph{atributov� uz�v�r} (mno�ina v�ech atribut� i tranzitivn� z�visl�ch na lev� stran�) -- to se opakovan� zkou��, jestli d�ky funk�n�m z�vislostem nedostanu z atribut� p�vodn� mno�iny n�jak� dal�� atributy (dokud nach�z�m dal��, p�id�v�m je do mno�iny a opakuji).
\end{poznamka}

\subsubsection*{Referen�n� integrita}

\begin{pitemize}
	\item pom�h� udr�ovat vztahy v rela�n� propojen�ch datab�zov�ch tabulk�ch, zabra�uje vzniku nekonzistentn�ch dat
	\item kontrola p��pustn�ch hodnot
	\item kontrola existence polo�ky s dan�m kl��em v druh� tabulce (podle ciz�ho kl��e) 
\end{pitemize}

Chov�n� p�i poru�en� integrity:
\begin{pitemize}
	\item ON UPDATE, ON DELETE - podm�nka spu�t�n� akce
	\item ON \dots RESTRICT - defaultn� �e�en� (hl�en� chyby)
	\item CASCADE - kask�dov� aktualizace/smaz�n� (sma�e p��slu�n� ��dky v odkazovan� tabulke)
	\item SET NULL - nastaven� odkazovan�ch ��dk� z�visl� tabulky na NULL
	\item SET DEFAULT - nastaven� pevn� ur�en� hodnoty
	\item NO ACTION 
\end{pitemize}

\subsection{Z�klady SQL}
TODO: p�evzato od \uv{program�tor�} z ot�zky \uv{SQL}, vzhledem k tomu, �e u n�s
se to jmenuje \uv{z�klady SQL} tak to mo�n� nemus� b�t tak podrobn�

Zdroje: slidy z p�edn�ek Datab�zov� syst�my a Datab�zov� aplikace Dr. T. Skopala a Dr. M. Kopeck�ho.

\subsubsection*{Standardy SQL}

SQL (\emph{Structured query language}) je standardn� jazyk pro p��stup k rela�n�m datab�z�m (a dotazov�n� nad nimi). Je z�rove� jazykem pro definici dat (definition data language), vytv��en� a modifikace sch�mat (tabulek), manipulaci s daty (data manipulation language), vkl��n�, aktualizace, maz�n� dat, ��zen� transakc�, definici integritn�ch omezen� aj. Jeho syntaxe odr�� snahu o co nejp�irozen�j�� formulace po�adavk� -- je podobn� anglick�m \uv{v�t�m}.

SQL je standard podle norem ANSI/ISO a existuje v n�kolika (zp�tn� kompatibiln�ch) verz�ch (ozna�ovan�ch podle roku uveden�):
\begin{description}
    \item[SQL 86] -- prvn� \uv{n�st�el}, pr�nik implementac� SQL firmy IBM
    \item[SQL 89] -- mal� revize motivovan� komer�n� sf�rou, mnoho detail� ponech�no implementaci
    \item[SQL 92] -- mnohem siln�j�� a obs�hlej�� jazyk. Zahrnuje u�
    \begin{pitemize}
	\item modifikace sch�mat, tabulky s metadaty, 
	\item vn�j�� spojen�, mno�inov� operace
	\item kask�dov� maz�n�/aktualizace podle ciz�ch kl���, transakce
	\item kurzory, v�jimky
    \end{pitemize}
    Standard existuje ve �ty�ech verz�ch: Entry, Transitional, Intermediate a Full.
    \item[SQL 1999] -- p�in�� mnoho nov�ch vlastnost�, nap�. 
    \begin{pitemize}	
	\item objektov�-rela�n� roz���en�
	\item nov� datov� typy -- reference, pole, full-text
	\item podpora pro extern� datov� soubory, multim�dia
	\item triggery, role, programovac� jazyk, regul�rn� v�razy, rekurzivn� dotazy ...
    \end{pitemize}
    \item[SQL 2003] -- dal�� roz���en�, nap�. XML management
\end{description}

Komer�n� syst�my implementuj� SQL podle r�zn�ch norem, n�kdy jenom SQL-92 Entry, dnes nej�ast�ji SQL-99, ale nikdy �pln� striktn�. N�kter� v�ci chyb� a naopak maj� v�echny spoustu nep�enositeln�ch roz���en� -- nap�. specifick� roz���en� pro procedur�ln�, transak�n� a dal�� funkcionalitu (T-SQL (Microsoft SQL Server), PL-SQL (Oracle) ). S nov�mi verzemi se kompatibilita zlep�uje, �asto je mo�n� pou��vat oboj� syntax. P�enos aplikace za b�hu na jinou platformu je ale st�le velice n�ro�n� -- a to t�m n�ro�n�j��, ��m v�c v�c� mimo SQL-92 Entry obsahuje.Pro otestov�n�, zda je �patn� syntax SQL, nebo zda jen dan� datab�zov� platforma nepodporuje n�kter� prvek, slou�� SQL valid�tory (kter� testuj� SQL podle norem.


\subsubsection*{Dotazy v SQL}

Hlavn�m n�strojem dotaz� v SQL je p��kaz \texttt{SELECT}. Sd�l� prvky rela�n�ho kalkulu i rela�n� algebry -- obsahuje pr�ci se sloupci, kvantifik�tory a agrega�n� funkce z rela�n�ho kalkulu a dal�� operace -- projekce, selekce, spojen�, mno�inov� operace -- z rela�n� algebry. Na rozd�l od striktn� formulace rela�n�ho modelu datab�ze povoluje duplik�tn� ��dky a NULLov� hodnoty atribut�.

Net��d�n� dotaz v SQL sest�v� z:
\begin{pitemize}
    \item p��kazu(�) \texttt{SELECT} (hlavn� logika dotazov�n�), to obsahuje v�dy
    \item m��e obsahovat i mno�inov� operace nad v�sledky p��kaz� \texttt{SELECT} -- \texttt{UNION}, \texttt{INTERSECTION} ...
\end{pitemize}
V�sledky nemaj� definovan� uspo��d�n� (resp. jejich po�ad� je ur�eno implementac� vyhodnocen� dotazu).

P��kaz \texttt{SELECT} vypad� n�sledovn� (tato verze u� zahrnuje i t��d�n� v�sledk�):
\begin{verbatim}
SELECT [DISTINCT]
 v�raz1 [[AS] c_alias1] [, ...]
FROM
 zdroj1 [[AS] t_alias1] [, ...]
[WHERE podm�nka_�]
[GROUP BY v�raz_g1 [, �]
[HAVING podm�nka_s]]
[ORDER BY v�raz_o1 [, �] ASC/DESC]
\end{verbatim}
Kde
\begin{pitemize}
    \item v�razy mohou b�t sloupce, sloupce s agrega�n�mi funkcemi, v�sledky dal��ch funkc� ...

\noindent \texttt{ v�raz = <n�zev sloupce>, <konstanta>, \\
 (DISTINCT) COUNT(~<n�zev sloupce>~),\\
{}[DISTINCT] [~SUM~|~AVG~](~<v�raz>~),\\
{}[~MIN~|~MAX~](~<v�raz>~)}\\
a nav�c lze pou��t oper�tory $+,-,*,/$.

    \item zdroje jsou tabulky nebo vno�en� selecty
    \item v�razy i zdroje b�t p�ejmenov�ny pomoc� \texttt{AS}, nap�. pro odkazov�n� uvnit� dotazu nebo jm�na na v�stupu (od SQL-92)
    \item podm�nka je logick� podm�nka (spojovan� logick�mi spojkami \texttt{AND, OR}) na hodnoty dat ve zdroj�ch:

\texttt{podm�nka = <v�raz> BETWEEN <x> AND <y>, <v�raz> LIKE "\%\_ ... ",\\
<v�raz> IS [NOT] NULL,\\
<v�raz> > = <> <= < > [<v�raz>/ ALL / ANY <dotaz>],\\
<v�raz> NOT IN [<seznam hodnot> / <dotaz>], EXIST ( <dotaz> )}

    \item \texttt{GROUP BY} znamen� agregaci podle unik�tn�ch hodnot jmenovan�ch sloupc� (v ostatn�ch sloupc�ch vznikaj� mno�iny hodnot, kter� se spolu s on�mi unik�tn�m� vyskytuj� na stejn�ch ��dk�ch
    \item \texttt{HAVING} ozna�uje podm�nku na agregaci
    \item \texttt{ORDER BY} definuje, podle hodnot ve kter�ch sloupc�ch nebo podle kter�ch jin�ch v�raz� nad nimi proveden�ch se m� v�sledek set��dit (\texttt{ASC} po�aduje vzestupn� set��d�n�, \texttt{DESC} sestupn�).
\end{pitemize}

SQL nem� p��kaz na omezen� rozsahu na n�kter� ��dky (jako nap�. \uv{pot�ebuji jen 50.-100. ��dek v�pisu}), a to lze �e�it bu� slo�it� standardn� (po��t�n� kolik hodnot je men��ch ne� vybran�, nav�c n�ro�n� na hardware) nebo pomoc� n�kter�ho nep�enositeln�ho roz���en�.

\medskip\noindent
Po�ad� vyhodnocov�n� jednoho p��kazu \texttt{SELECT} (nebereme v �vahu optimalizace):
\begin{penumerate}
    \item Nejprve se zkombinuj� data ze v�ech zdroj� (tabulek, pohled�, poddotaz�). Pokud jsou odd�leny ��rkami, provede se kart�zsk� sou�in (to sam� co \texttt{CROSS JOIN}), v SQL-92 a vy���m i slo�it�j�� spojen� -- \texttt{JOIN ON} (vnit�n� spojen� podle podm�nky), \texttt{NATURAL JOIN} (\uv{p�irozen�} spojen� podle stejn�ch hodnot stejn� pojmenovan�ch sloupc�), \texttt{OUTER JOIN} (\uv{vn�j��} spojen�, do kter�ho jsou zahrnuty i z�znamy, pro kter� v jednom ze zdroj� nen� nalezeno nic, co by odpov�dalo podm�nce, dopln�nn� NULLov�mi hodnotami) atd.
    \item Vy�ad� se vznikl� ��dky, kter� nevyhovuj� podm�nce (\texttt{WHERE})
    \item Zbyl� ��dky se seskup� do skupin se stejn�mi hodnotami uveden�ch v�raz� (\texttt{HAVING}), ka�d� skupina obsahuje atomick� sloupce s hodnotami uveden�ch v�raz� a mno�inov� sloupce se skupinami ostatn�ch hodnot sloupc�.
    \item Vy�ad� se skupiny, nevyhovuj�c� podm�nce (\texttt{HAVING})
    \item V�sledky se set��d� podle po�adavk�
    \item Vygeneruje se v�stup s po�adovan�mi hodnotami
    \item V p��pad� \texttt{DISTINCT} se vy�ad� duplicitn� ��dky
\end{penumerate}


\begin{e}{Pozn�mka}{0}{0}
\begin{pitemize}
    \item Klauzule \texttt{GROUP BY} set��d� p�ed vytvo�en�m skupin v�echny ��dky dle v�raz� v klauzuli. Proto by se m�l seskupovat co nejmen�� mo�n� po�et ��dek. Pokud je mo�n� ��dky odfiltrovat pomoc� WHERE, je v�sledek efektivn�j��, ne� n�sledn� odstra�ov�n� cel�ch skupin.
    \item  Klauzule \texttt{DISTINCT} t��d� v�sledn� z�znamy (p�ed operac� ORDER BY), aby na�la duplicitn� z�znamy. Pokud to jde, je vhodn� se bez n� obej�t.
    \item Klauzule \texttt{ORDER BY} by m�la b�t pou�ita jen v nutn�ch p��padech. Nen� p��li� vhodn� ji pou��vat v definic�ch pohled�, nad kter�mi se d�le d�laj� dal�� dotazy
\end{pitemize}
\end{e}


\subsubsection*{Definice a manipulace s daty, ostatn� p��kazy}

Standard SQL podporuje n�kolik druh� datov�ch typ�:
\begin{pitemize}
    \item textov� v n�rodn� a glob�ln� (UTF) znakov� sad� (n�kolika druh� -- prom�nn� a pevn� d�lky): \texttt{CHARACTER(n)}, \texttt{NCHAR(n)},
    \texttt{CHAR VARYING(n)}
    \item ��seln� typy -- \texttt{ NUMERIC(p[,s]), INTEGER, INT, SMALLINT,\\  FLOAT(presnost), REAL, DOUBLE PRECISION}
    \item datumov� typy -- \texttt{DATE, TIME, TIMESTAMP, TIMESTAMP(presnost\_sekund) WITH TIMEZONE}
\end{pitemize}
Datab�zov� servery ne v�dy podporuj� v�echny uveden� typy. Nemus� je podporovat nativn�, n�kdy si pouze \uv{p�elo��} n�zev typu na podobn� nativn� podporovan� typ.

\medskip
\begin{obecne}{P��kaz \texttt{CREATE TABLE}}
Tento p��kaz slou�� k vytvo�en� nov� tabulky. Je nutn� definovat jej� n�zev, atributy a jejich dom�ny (datov� typy); d�le je mo6n� definovat integritn� omezen� (kl��e, ciz� kl��e, odkazy, podm�nky). P��kaz vypad� n�sledovn�:
\begin{center}
\texttt{CREATE TABLE <n�zev> <def. sloupce/i.o. tabulky, ...> }
\end{center}
A uvnit� potom
\begin{verbatim}
def. sloupce = <n�zev> <dat.typ> 
    [DEFAULT NULL|<hodnota>] [<i.o.sloupce>] 
dat.typ = [VARCHAR(n) | BIT(n) | INTEGER | FLOAT | DECIMAL ...] 
i.o.sloupce = [CONSTRAINT <jm�no>] [NOT NULL / UNIQUE / PRIMARY KEY], 
    REFERERENCES <tabulka>(<sloupec>) <akce>, CHECK <podm�nka> 
akce = [ON UPDATE / ON DELETE] 
    [CASCADE / SET NULL / SET DEFAULT / NO ACTION(hl�en� chyby) ] 
i.o.tabulky = UNIQUE, PRIMARY KEY <sloupec, ... >, 
    FOREIGN KEY <sloupec, ... >, 
    REFERENCES <tabulka>(<sloupec, ... >), 
    CHECK( <podm�nka> )
\end{verbatim}
\end{obecne}

\medskip
\begin{obecne}{P��kazy pro manipulaci se sch�matem}
\begin{pitemize}
    \item �prava tabulky:
\begin{verbatim}
ALTER TABLE <n�zev> ADD {COLUMN} <def.sloupce>, ADD <i.o.tabulky>, 
    ALTER COLUMN <sloupec> [ SET / DROP ], DROP COLUMN <sloupec>, 
    DROP CONSTRAINT <jm�no i.o.> 
\end{verbatim}
    \item Smaz�n� tabulky (nen� to sam� jako vymaz�n� v�ech dat z tabulky!):
\begin{verbatim}
DROP TABLE <tabulka> 
\end{verbatim}
    \item Vytvo�en� \uv{pohledu} -- navenek se chov� jako tabulka, ale vnit�n� se p�i ka�d�m dotazu provede vno�en� dotaz (kter� definic� pohledu zapisuji):
\begin{verbatim}
CREATE VIEW <n�zev "tabulky"> ( <sloupec, ... > ) 
    AS <dotaz> {WITH [ LOCAL / CASCADED ] CHECK OPTION }
\end{verbatim}
    N�kter� datab�zov� platformy umo��uj� do takto vytvo�en�ch pohled� i zapisovat.
\end{pitemize}
\end{obecne}

\medskip
\begin{obecne}{P��kazy pro manipulaci s daty}
\begin{pitemize}
    \item Vlo�en� nov�ch dat do tabulky
\begin{verbatim}
INSERT INTO <tabulka> ( <sloupec, ... > ) 
    [VALUES ( <v�raz, ... > ) / (<dotaz>) ] 
\end{verbatim}
    \item �prava dat (na ��dc�ch kter� vyhovuj� podm�nce se nastav� zadan� hodnoty vybran�m sloupc�m):
\begin{verbatim}
UPDATE <tabulka> SET 
    ( <sloupec> = [ NULL / <v�raz> / <dotaz> ] , ... ) 
    WHERE (<podm�nka>) 
\end{verbatim}
    \item Smaz�n� ��dk� vyhovuj�c�ch podm�nce z tabulky:
\begin{verbatim}
DELETE FROM <tabulka> ( WHERE <podm�nka> ) 
\end{verbatim}
\end{pitemize}
\end{obecne}

\subsection{Transak�n� zpracov�n�, vlastnosti transakc�, uzamykac� protokoly, zablokov�n�}

\begin{e}{Definice}{0}{Transakce}
\emph{Transakce} je jist� posloupnost nebo specifikace posloupnosti akc� pr�ce s datab�z�, jako
jsou �ten�, z�pis nebo v�po�et, se kterou se zach�z� jako s jedn�m celkem.
\end{e}

Hlavn�m smyslem pou��v�n� transakc�, tj. \emph{transak�n�ho zpracov�n�}, je
udr�en� datab�ze v konzistentn�m stavu. Jestli�e na sob� n�kter� operace z�vis�,
sdru��me je do jedn� transakce a t�m zabezpe��me, �e budou vykon�ny bu�
v�echny, nebo ��dn�. Datab�ze tak p�ed i po vykon�n� transakce bude v
konzistentn�m stavu. Aby se u�ivateli transakce jevila jako jedna atomick�
operace, je nutn� zav�st p��kazy COMMIT a ROLLBACK. Prvn� z nich signalizuje
datab�zi �sp�nost proveden� transakce, tj. ve�ker� zm�ny v datab�zi se stanou
trval�mi a jsou zviditeln�ny pro ostatn� transakce, druh� p��kaz signalizuje
opak, tj. datab�ze mus� b�t uvedena do p�vodn�ho stavu.

Tyto p��kazy v�t�inou nen� nutn� volat explicitn�, nap�. p��kaz COMMIT je vyvol�n po
norm�ln�m ukon�en� programu realizuj�c�ho transakci. P��kaz ROLLBACK pro svou
funkci vy�aduje pou�it� tzv. \emph{�urn�lu} (logu) na n�jak�m stabiln�m
pam�ov�m m�diu. �urn�l obsahuje historii v�ech zm�n datab�ze v jist� �asov�
period�.

Jednoduch� transakce vypad� v�t�inou takto:
\begin{penumerate}
  \item Za��tek transakce,
  \item proveden� n�kolika dotaz� -- �ten� a z�pis� (��dn� zm�ny v datab�zi nejsou zat�m vid�t pro
  okoln� sv�t),
  \item Potvrzen� (p��kaz COMMIT) transakce (pokud se transakce povedla, zm�ny
  v datab�zi se stanou viditeln�).
\end{penumerate}
Pokud n�jak� z proveden�ch dotaz� sel�e, syst�m by m�l celou transakci zru�it a
vr�tit datab�zi do stavu v jak�m byla p�ed zah�jen�m transakce (operace ROLLBACK).

Transak�n� zpracov�n� je tak� ochrana datab�ze p�ed hardwarov�mi nebo
softwarov�mi chybami, kter� mohou zanechat datab�zi po ��ste�n�m zpracov�n�
transakce v nekonzistentn�m stavu. Pokud po��ta� sel�e uprost�ed prov�d�n�
n�kter� transakce, transak�n� zpracov�n� zaru��, �e v�echny operace z
nepotvrzen�ch (\uv{uncommitted}) transakc� budou zru�eny. 

\subsubsection*{Vlastnosti transakc�}

Pod�vejme se nyn� na vlastnosti po�adovan� po transakc�ch. Obvykle se pou��v�
zkratka prvn�ch p�smen anglick�ch n�zv� vlastnost� \textbf{ACID}~-- atomicity,
consistency, isolation (independence), durability. 
\begin{description}
  \item[atomicita] -- transakce se tv��� jako jeden celek, mus� bu� prob�hnout
  cel�, nebo v�bec ne.
  \item[konzistence] -- transakce transformuje datab�zi z jednoho konzistentn�ho
  stavu do jin�ho konzistentn�ho stavu.
  \item[nez�vislost] -- transakce jsou nez�visl�, tj. d�l�� efekty transakce
  nejsou viditeln� jin�m transakc�m.
  \item[trvanlivost] -- efekty �sp�n� ukon�en� (potvrzen�,\uv{commited})
  transakce jsou nevratn� ulo�eny do datab�ze a nemohou b�t zru�eny.
\end{description}

Transakce mohou b�t v u�ivatelsk�ch programech prov�d�ny paraleln� (sp�e
zd�nliv� paraleln�, stejn� jako je paralelismus multitaskingu na jednoprocesorov�ch
stroj�ch jen zd�nliv�, zajist� to ale mo�nost paralelizace \uv{nedatab�zov�ch} 
akc� a pomal� transakce nebrzd� rychl�). Je
z�ejm�, �e posloupnost transakc� m��e b�t zpracov�na paraleln� r�zn�m zp�sobem.
Ka�d� transakce se skl�d� z n�kolika akc�. Stanoven� po�ad� prov�d�n� akc�
v�ce transakc� v �ase nazveme \textbf{rozvrhem}.

Rozvrh, kter� spl�uje n�sleduj�c� podm�nky, budeme naz�vat \textbf{leg�ln�}:
\begin{pitemize}
  \item Objekt je nutn� m�t uzamknut�, pokud k n�mu chce transakce p�istupovat.
  \item Transakce se nebude pokou�et uzamknout objekt ji� uzamknut� jinou
  transakc� (nebo mus� po�kat, ne� bude objekt odemknut).
\end{pitemize}

D�le�it�mi pojmy pro paraleln� zpracov�n� jsou s�riovost �i uspo��datelnost.
\textbf{S�riov� rozvrhy} zachov�vaj� operace ka�d� transakce pohromad� (a 
prov�d� se jen jedna transakce najednou). Pro $n$
transakc� tedy existuje $n!$ r�zn�ch s�riov�ch rozvrh�. Pro z�sk�n� korektn�ho
v�sledku v�ak m��eme pou��t i rozvrhu, kde jsou operace r�zn�ch transakc�
navz�jem prokl�d�ny.
P�irozen�m po�adavkem na korektnost je, aby efekt paraleln�ho zpracov�n�
transakc� byl t��, jako kdyby transakce byly provedeny v n�jak�m s�riov�m rozvrhu.
P�edpokl�d�me-li toti�, �e ka�d� transakce je korektn� program, m�l by v�st
v�sledek s�riov�ho zpracov�n� ke konzistentn�mu stavu. O syst�mu zpracov�n�
transakc�, kter� zaru�uje dosa�en� konzistentn�ho stavu nebo stejn�ho stavu
jako s�riov� rozvrhy, se ��k�, �e zaru�uje \textbf{uspo��datelnost}.

Mohou se vyskytnout probl�my, kter� uspo��datelnosti zamezuj�. Ty naz�v�me \emph{konflikty}. Plynou z po�ad� dvojic akc� r�zn�ch transakc� na stejn�m objektu. Existuj� t�i typy konfliktn�ch situac�:
\begin{penumerate}
    \item WRITE-WRITE -- p�eps�n� nepotvrzen�ch dat
    \item READ-WRITE -- neopakovateln� �ten�
    \item WRITE-READ -- �ten� nepotvrzen�ch (\uv{uncommitted}) dat
\end{penumerate}

�ekneme, �e rozvrh je \emph{konfliktov� uspo��dateln�}, je-li konfliktov� ekvivalentn� n�jak�mu s�riov�mu rozvrhu (tedy jsou v n�m stejn�, tj. ��dn� konflikty). Test na konfliktovou uspo��datelnost se d� prov�st jako test acykli�nosti grafu, ve kter�m konfliktn� situace p�edstavuj� hrany a transakce vrcholy. Konfliktov� uspo��datelnost je slab�� podm�nka ne� uspo��datelnost -- nezohled�uje ROLLBACK (\emph{zotavitelnost} -- zachov�n� konzistence, i kdy� kter�koliv transakce sel�e) a dynamickou povahu datab�ze (vkl�d�n� a maz�n� objekt�). Zotavitelnosti se d� dos�hnout tak, �e ka�d� transakce $T$ je potvrzena a� pot�, co jsou potvrzeny v�echny ostatn� transakce, kter� zm�nily data �ten� v $T$. Pokud v zotaviteln�m rozvrhu doch�z� ke �ten� zm�n pouze potvrzen�ch transakc�, nem��e doj�t ani k jejich \emph{kask�dov�mu ru�en�}.

P�i zpracov�n� (i uspo��dateln�ho) rozvrhu m��e doj�t k situaci \emph{uv�znut�} -- \emph{deadlocku}. To nastane tehdy, pokud jedna transakce $T_1$ �ek� na z�mek na objekt, kter� m� p�id�len� $T_2$ a naopak. Situaci lze zobecnit i na v�ce transakc�. Uv�znut� lze bu� p��mo zamezit charakterem rozvrhu, nebo detekovat (hled�n�m cyklu v grafu �ekaj�c�ch transakc�, tzv. \uv{waits-for} grafu) a jednu z transakc� \uv{zab�t} a spustit znova.

\medskip
K zaji�t�n� uspo��datelnosti a zotavitelnosti a zabezpe�en� proti kask�dov�m rollback�m a deadlocku se pou��vaj� r�zn� sch�mata (po�adavky na rozvrhy). Jedn�m z nich jsou uzamykac� protokoly.

\subsubsection*{Uzamykac� protokoly}

Vytv��en� rozvrh� a testov�n� jejich uspo��datelnosti nen� pro praxi z�ejm� ten
nejvhodn�j�� zp�sob. Pokud ale budeme transakce konstruovat podle ur�it�ch
pravidel, tak za ur�it�ch p�edpoklad� bude ka�d� jejich rozvrh uspo��dateln�.
Soustav� takov�ch pravidel se ��k� \textbf{protokol}.

Nejzn�m�j�� protokoly jsou zalo�eny na dynamick�m zamyk�n� a odemyk�n� objekt� v
datab�zi. Zamyk�n� (operace LOCK) je akce, kterou vyvol� transakce na objektu,
aby ho chr�nila p�ed p��stupem ostatn�ch transakc�.

\begin{e}{Definice}{0}{Dob�e formovan� transakce}
Transakci nazveme \textbf{dob�e formovanou} pokud podporuje p�irozen� po�adavky
na transakce:
\begin{penumerate}
  \item transakce zamyk� objekt, chce-li k n�mu p�istupovat,
  \item transakce nezamyk� objekt, kter� ji� je touto transakc� uzam�en�,
  \item transakce neodmyk� objekt, kter� nen� touto transakc� zam�en�,
  \item po ukon�en� transakce jsou v�echny objekty uzam�en� touto transakc�
  odem�eny.
\end{penumerate}
\end{e}

\paragraph{Dvouf�zov� protokol (2PL)} -- Dvouf�zov� transakce v prvn� f�zi
zamyk� v�e co je pot�eba a od prvn�ho odemknut� (druh� f�ze) ji� jen odemyk� co
m�la zam�eno (ji� ��dn� operace LOCK). Tedy transakce mus� m�t v�echny objekty
uzam�eny p�edt�m, ne� n�jak� objekt odemkne. D� se dok�zat, �e pokud jsou
v�echny transakce v dan� mno�in� transakc� dob�e formovan� a dvouf�zov�, pak
ka�d� jejich leg�ln� rozvrh je uspo��dateln�.

Dvouf�zov� protokol zaji��uje uspo��datelnost, ale ne zotavitelnost ani
bezpe�nost proti kask�dov�mu ru�en� transakc� nebo uv�znut�.

\paragraph{Striktn� dvouf�zov� protokol (S2PL)} -- Probl�my 2PL jsou nezotavitelnost
a kask�dov� ru�en� transakc�. Tyto nedostatky lze odstranit pomoc� striktn�ch
dvouf�zov�ch protokol�, kter� uvol�uj� z�mky a� po skon�en� transakce (COMMIT).
Z�ejm� nev�hoda je omezen� paralelismu. 2PL nav�c st�le nevylu�uje mo�nost deadlocku.

\paragraph{Konzervativn� dvouf�zov� protokol (C2PL)} -- Rozd�l oproti 2PL je
ten, �e transakce ��d� o v�echny sv� z�mky, je�t� ne� se za�ne
vykon�vat. To sice vede ob�as k zbyte�n�mu zamyk�n� (nev�me co p�esn� budeme
pot�ebovat, tak rad�i zamkneme v�c), ale sta�� to ji� k prevenci uv�znut�
(deadlocku).

\subsubsection*{\uv{Vylep�en�} zamykac�ch protokol�}

\paragraph{Sd�len� a v�lu�n� z�mky} -- Nev�hodou 2PL je, �e objekt m��e m�t
uzam�en� pouze jedna transakce. Abychom uzamyk�n� provedli precizn�ji, je dobr�
vz�t na v�dom� rozd�l mezi operacemi READ a WRITE. \emph{V�lu�n� z�mek}
(W\_LOCK) m��e b�t aplikov�n na objekty jak pro operaci READ tak pro WRITE,
\emph{sd�len� z�mek} (R\_LOCK) uzamyk� objekt, kter� chceme pouze ��st. Jeden
objekt potom m��e b�t uzam�en sd�len�m z�mkem v�ce transakc� a zvy�uje se tak
mo�nost paraleln�ho zpracov�n�. Budeme-li s t�mito z�mky zach�zet stejn� jako u
2PL, op�t m�me zaru�enou uspo��datelnost rozvrhu, ov�em nikoliv absenci uv�znut�.


\paragraph{Strukturovan� uzamyk�n� (multiple granularity)} -- Objekty jsou v
tomto p��pad� ch�p�ny hierarchicky dle relace \emph{obsahuje}. Nap��klad
datab�ze obsahuje soubory, kter� obsahuj� str�nky a ty zase obsahuj� jednotliv�
z�znamy. Na tuto hierarchii se m��eme d�vat jako na strom, ve kter�m ka�d�
vrchol obsahuje sv� potomky. Kdy� transakce zamyk� objekt (vrchol) zamyk� tak�
v�echny jeho potomky. Protokol se tak sna�� minimalizovat po�et z�mk�, t�m
sn�it re�ii a zv��it mo�nosti paraleln�ho zpracov�n�.


\subsubsection*{Alternativn� protokoly}

\paragraph{�asov� raz�tka} -- Dal�� z protokol� zaru�uj�c� uspo��datelnost je
vyu�it� �asov�ch raz�tek. Na za��tku dostane transakce $T$ \emph{�asov�
raz�tko}~-- $TS(T)$ (�asov� raz�tka jsou unik�tn� a v �ase rostou), abychom v�d�li
po�ad�, ve kter�m by m�li b�t transakce vykon�ny. Ka�d� objekt v datab�zi m�
\emph{�tec� raz�tko}~-- $RTS(O)$ (read timestamp), kter� je aktualizov�no, kdy� je
objekt �ten, a \emph{zapisovac� raz�tko}~-- $WTS(O)$ (write timestamp), kter� je
aktualizov�no, kdy� n�jak� transakce objekt m�n�.

Pokud chce transakce $T$ ��st objekt $O$ mohou nastat dva p��pady:
\begin{pitemize}

  \item $TS(T) < WTS(O)$, tzn. n�kdo zm�nil objekt $O$ potom co byla spu�t�na
  transakce $T$. V tomto p��pad� mus� b�t transakce zru�ena a spou�t�na znovu (a
  tedy s jin�m �asov�m raz�tkem).

  \item $TS(T) > WTS(O)$, tzn. je bezpe�n� objekt ��st. V tomto p��pad� $T$
  p�e�te $O$ a $RTS(O)$ je nastaveno na $\max\{TS(T),\ RTS(O)\}$.

\end{pitemize}

Pokud chce transakce $T$ zapisovat do objektu $O$ rozli�ujeme p��pady t�i:
\begin{pitemize}

  \item $TS(T) < RTS(O)$, tzn. n�kdo �etl $O$ pot� co byla spu�t�na $T$ a
  p�edpokl�d�me, �e si po��dil lok�ln� kopii. Nem��eme tedy $O$ zm�nit, proto�e
  by lok�ln� kopie p�estala b�t platn� a tedy je nutn� $T$ zru�it a spustit
  znova.

  \item $TS(T) < WTS(O)$, tzn. n�kdo zm�nil $O$ po startu $T$. V tomto p��pad�
  p�esko��me write operaci a pokra�ujeme d�le norm�ln�. $T$ nemus� b�t
  restartov�na.

  \item V ostatn�ch p��padech $T$ zm�n� $O$ a $WTS(O)$ je nastaveno na $TS(T)$.
\end{pitemize}

\paragraph{Optimistick� protokoly} -- V situaci kdy se v�t�ina transakc�
neovliv�uje, je re�ie v��e uveden�ch protokol� zbyte�n� velk� a m��eme pou��t
takzvan� optimistick� protokol. V protokolu m��eme rozli�it t�i f�ze.
\begin{penumerate}

  \item \textbf{F�ze �ten�:} �tou se objekty z datab�ze do lok�ln� pam�ti a jsou
  na nich prov�d�ny pot�ebn� zm�ny.

  \item \textbf{F�ze kontroly:} Po dokon�en� v�ech zm�n v lok�ln� pam�ti je
  vyvol�n pokus o zaps�n� v�sledk� do datab�ze. Algoritmus zkontroluje, zda
  nehroz� potenci�ln� kolize s ji� potvrzen�mi transakcemi, nebo s n�kter�mi
  pr�v� prob�haj�c�mi. Pokud konflikt existuje, je t�eba spustit algoritmus pro
  �e�en� koliz�, kter� se je sna�� vy�e�it. Pokud se mu to nepoda��, je vyu�ita
  posledn� mo�nost a tou je zru�en� a restartov�n� transakce.

  \item \textbf{F�ze z�pisu:} Pokud nehroz� ��dn� konflikty, jsou data z lok�ln�
  pam�ti zaps�ny do datab�ze a transakce potvrzena.

\end{penumerate}




\subsection{Organizace dat na vn�j�� pam�ti, B-stromy a jejich varianty}


\subsubsection*{Vn�j�� pam�}

\begin{e}{Definice}{0}{Vn�j�� pam�}
\emph{Vn�j�� pam�} je �lo�i�t� dat (pam�ov� m�dium), u kter�ho je rychlost na��t�n� dat zpravidla n�zk� a p��stup k nim ne �pln� p��m� (z�le�� na uspo��d�n� dat na m�diu), ne-li pouze sekven�n� (oproti vnit�n� pam�ti s rychl�m n�hodn�m p��stupem a men�� kapacitou). P��kladem vn�j�� pam�ti je pevn� disk nebo magnetick� p�ska.

\emph{Magnetick� p�sky} poskytuj� vysokou kapacitu, ale n�zkou rychlost a pouze sekven�n� p��stup. Pro jejich kapacitu je d�le�it� hustota z�znamu, pot�ebuj� meziblokov� mezery pro vyrovn�n� nep�esnosti p�et��en� p�sky.

\emph{Pevn� disky} umo��uj� p��m� p��stup, ale jeho rychlost nen� v�dy stejn�. Ovliv�uje ji fyzick� vzd�lenost dat -- pevn� disk m� n�kolik \emph{v�lc�}, na nich� jsou ulo�eny jednotliv� datov� \emph{stopy}. K v�lc�m p��slu�� \emph{�tec� hlavy} (je jich stejn� jako v�lc�, ale pohybuj� se v�echny sou�asn�, tak�e m��e v 1 okam�iku ��st jen jedna). Disky jsou v�t�inou rozd�leny na \emph{sektory} -- nejmen�� jednotku dat, kterou je mo�n� na��st nebo ulo�it (zpravidla jednotky KB). Pro rychlost p��stupu k dat�m jsou d�le�it� tyto veli�iny (v�robcem disk� jsou zpravidla ud�v�ny pr�m�rn� hodnoty):
\begin{pitemize}
    \item \emph{seek} ($s$) -- p�esun na jinou stopu, dnes zpravidla kolem 4-8~ms
    \item \emph{(average) rotational delay} ($r$) -- oto�en� v�lc� -- 1 p�lot��ka, pro nej�ast�j�� 7200rpm disk je to cca 4~ms
    \item \emph{block transfer time} ($btt$) -- doba p�enesen� 1 bloku po sb�rnici, na ATA/100 disku se 4~KB bloky teoreticky 0.04~ms
\end{pitemize}
Pokud jsou data um�st�na na disku za sebou sekven�n�, rychlost jejich na�ten� je mnohem vy��� ne� p�i n�hodn�m rozm�st�n�, proto�e nen� t�eba prov�d�t p�esuny mezi stopami a ot��en� v�lc� nav�c.
\end{e}

\begin{e}{P��klad}{0}{0}
\emph{Jak vypad� na�ten� 1~MB dat z pevn�ho disku?} P�edpokl�dejme, �e na 1 stopu se vejde 512~KB a 1 blok m� 4~KB. Jsou-li data um�st�na na disku sekven�n�, pot�ebuji pro na�ten� 1~MB dat naj�t prvn� blok a potom ��st dv� cel� stopy (2 ot��ky), tj. celkem $s+r+(2\cdot2r)$ (a p�enos po sb�rnici lze zanedbat, proto�e prob�h� z�rove� se �ten�m). Pokud jsou data na disku n�hodn� rozprost�ena, pot�ebuji celkem 256-kr�t naj�t blok a na��st ho: $256\cdot(s+r+btt)$, tak�e operace trv� a� 100-kr�t d�le.
\end{e}


\subsubsection*{Soubor}

\begin{e}{Definice}{0}{Z�znam, kl��}
\emph{(Logick�) z�znam} je jednotka dat (nap�. v datab�zi), m� \emph{atributy} (z nich� ka�d� m� jm�no a dom�nu -- povolenou mno�inu hodnot). Logick�mu z�znamu v reprezentaci na disku odpov�d� \emph{fyzick� z�znam} (n�jak� d�lky $R$ -- pevn� nebo prom�nn�), kter� nav�c m��e obsahovat je�t� dal�� data -- odd�lova�e, ukazatele, hlavi�ky. 

\emph{Kl��} je mno�ina atribut�, kter� jednozna�n� identifikuje z�znam; proti tomu \emph{vyhled�vac� kl��} je mno�ina atribut�, pro kterou lze nal�zt mno�inu odpov�daj�c�ch z�znam�. Vyhled�vac� kl��e jsou t�� druh�: hodnotov� (\uv{oby�ejn�} hodnoty n�kter�ch atribut�), ha�ovan� a relativn� (p��mo pozice v souboru).
\end{e}

\begin{e}{Definice}{0}{Soubor}
(Homogenn�) \emph{soubor} je multimno�ina z�znam�. Fyzicky na vn�j�� pam�ti je organizov�n do \emph{blok� (str�nek)} (velikosti $B$, typicky n�kolika KB) -- hl. jednotkou p�enosu dat mezi vnit�n� a vn�j�� pam�t�. Pom�r velikosti z�znamu k velikosti bloku ($B/R$) se naz�v� \emph{blokovac� faktor} ($\lfloor b\rfloor$). Z�znamy mohou b�t rozprost�eny i p�es n�kolik str�nek, nebo m��e b�t pouze jeden z�znam na 1 str�nku; ide�ln� (ale ne v�dy dosa�iteln�) je, pokud beze zbytku zapl�uj� str�nky. Na souboru jsou definov�ny operace se z�znamy: \emph{insert, delete, update} a \emph{fetch}.
\end{e}

\begin{e}{Definice}{0}{Dotaz}
\emph{Dotaz} je ka�d� funkce, kter� ka�d�mu sv�mu argumentu p�i�ad� odpov�daj�c� mno�inu z�znam� ze souboru (\uv{tot�ln� vy��sliteln� funkce definovan� na souboru}). Dotazy mohou b�t t�chto typ�:
\begin{pitemize}
    \item Na�ten� v�ech z�znam� ({\tt SELECT * FROM tabulka})
    \item Na �plnou shodu ({\tt SELECT * FROM tabulka WHERE sloupec1 = 'hodnota' AND sloupec2 = 'hodnota'} pro tabulku se 2 sloupci -- d�ny jsou v�echny atributy)
    \item Na ��ste�nou shodu\\({\tt SELECT * FROM tabulka WHERE sloupec1 = 'hodnota'} pro tabulku se 2 sloupci -- zadan� je jen ��st atribut�)
    \item Na intervalovou shodu (��ste�nou nebo �plnou) ({\tt SELECT * FROM tabulka WHERE sloupec1 > 'hodnota'})
\end{pitemize}
U soubor� se sleduje rychlost proveden� t�chto operac�.
\end{e}


\subsubsection*{Statick� metody organizace souboru}

\begin{e}{Definice}{0}{Sch�ma organizace souboru}
\emph{Sch�ma organizace souboru} je popis logick� pam�ov� struktury, do n� lze zobrazit logick� soubor, spolu s algoritmy operac� nad touto strukturou. Ta je obvykle tvo�ena z logick�ch str�nek (blok� pevn� d�lky) a m��e popisovat v�ce prov�zan�ch log. soubor�, z nich� \emph{prim�rn� soubor} je ten, kter� obsahuje u�ivatelsk� data. Operace definovan� nad sch�matem org. souboru jsou krom� operac� nad soubory je�t� \emph{build, reorganization, open} a \emph{close}.

Proti n�mu stoj� \emph{fyzick� sch�ma souboru} -- struktura nad fyzick�mi soubory, nejbl�e hardwaru je \emph{implementa�n� sch�ma souboru}.

Zaji�t�n� \emph{Vyv�enosti struktury} znamen� zaji�t�n� omezen� cesty p�i vyhled�v�n� n�jak�m v�razem (zaru�en� asymptotick� slo�itosti), nav�c zaru�en� rovnom�rnosti zapln�n� struktury -- \emph{faktor napln�n� str�nek}. Sch�mata, kter� spl�uj� ob� podm�nky, se naz�vaj� \emph{dynamick�}, ostatn� jsou ozna�ov�na jako \emph{statick�}.
\end{e}

\begin{e}{Pozn�mka}{0}{�asov� odhady}
Pro sch�mata organizace soubor� se po��taj� �asov� odhady proveden� jednotliv�ch operac� -- jednodu���ch, jako je p��stup k z�znamu ($T_F$), \emph{rewrite} -- p�epis b�hem 1 ot��ky disku ($T_{RW}$), p��p. sekven�n� �ten�; d�le i slo�it�j��ch jako vyhled�n� z�znamu, p�id�n�, smaz�n� a �prava z�znamu, reorganizace struktury nebo na�ten� cel�ho souboru.
\end{e}

\begin{obecne}{Hromada (neuspo��dan� sekven�n� soubor)}
\emph{Hromada(heap)} je naprosto nejjednodu��� sch�ma organizace souboru, kdy jsou z�znamy v souboru jen n�hodn� se�azeny za sebou. �asov� slo�itost vyhled�v�n� je $O(n)$, pokud $n$ je po�et z�znam�. Jde o \emph{nehomogenn� soubor}, kde z�znamy obvykle nemaj� pevnou d�lku. 
\end{obecne}

\begin{obecne}{Uspo��dan� sekven�n� soubor}
V \emph{uspo��dan�m sekven�n�m souboru} jsou z�znamy �azeny podle kl��e. Aktualizovan� z�znamy se um�st� do zvl�tn�ho souboru a a� p�i dal�� operaci \uv{reorganization} jsou p�id�ny do prim�rn�ho. Slo�itost nalezen� z�znamu je tak� $O(n)$, ale pokud se hled� podle kl��e, podle kter�ho jsou z�znamy se�azeny, a nav�c je soubor na m�diu s p��m�m p��stupem, sn�� se na $O(\log n)$.
\end{obecne}

\begin{obecne}{Index-sekven�n� soubor}
Toto sch�ma uva�uje prim�rn� soubor jako sekven�n�, uspo��dan� podle prim�rn�ho kl��e. Nad n�m je pak vytvo�en (t�eba i v�ce�rov�ov�) \emph{index}. Ten sest�v� ze seznamu ��sel str�nek a minim�ln�ch hodnot kl��e jim odpov�daj�c�ch z�znam�. Pokud m� index v�c �rovn�, prov�d� se pro vy��� �rovn� to sam� na bloc�ch index� �rovn� o 1 ni���. Nejvy��� �rove� indexu se obvykle vejde do 1 bloku, tzv. \emph{master}. 

Po�et pot�ebn�ch �rovn� pro $n$ z�znam� se d� spo��tat jako $\lceil\log_p\frac{n}{\lfloor b\rfloor}\rceil$, kde $p=\lfloor\frac{B}{V+P}\rfloor$ p�i velikosti kl��e $V$ a pointeru na str�nku $P$. Probl�mem je p�id�v�n� nov�ch z�znam�, kdy se tyto �et�z� za sebe v tzv. \emph{oblasti p�ete�en�} (ka�d� z nich m� pointer na dal�� z�znam v oblasti p�ete�en�). Pro odd�len� nutnosti vkl�d�n� do oblasti p�ete�en� lze inici�ln� bloky plnit na m�n� ne� 100\%.
\end{obecne}


\begin{obecne}{Indexovan� soubor}
\emph{Indexovan� soubor} znamen� prim�rn� soubor plus indexy pro r�zn� vyhled�vac� kl��e. Neindexuj� se u� str�nky, ale p��mo z�znamy, a proto prim�rn� soubor nemus� b�t nutn� set��d�n�. Index m��e b�t podobn� jako u index-sekven�n�ho souboru, pro z�znamy se stejn�m kl��em je ale vhodn�, aby byly na v�ech �rovn�ch indexu krom� posledn� slou�en�. P�i aktualizaci se nepou��v� oblast p�ete�en�, m�n� se pouze index.

Existuje i n�kolik dal��ch variant index�. Pro zmen�en� n�ro�nosti dotaz� na kombinovanou ��ste�nou shodu se pou��v� \emph{kombinovan� index} pro v�ce atribut�, u n�ho� je ale nutn� p�edem zjistit na kter� kombinace atribut� budou �asto pokl�d�ny dotazy, a pro takov� kombinace tento index teprve vytvo�it. \emph{Clusterovan� index} zaru�uje, �e z�znamy s podobnou hodnotou indexovan�ho atributu jsou bl�zko sebe v prim�rn�m souboru -- nap�. pokud je prim�rn� soubor podle tohoto atributu set��d�ny. Tento index lze pou��t jen pro 1 atribut. 

\emph{Bitov� mapy} se daj� pou��t jako index pro atributy s malou dom�nou (mno�inou mo�n�ch hodnot) -- pro ka�dou hodnotu t�to dom�ny se vyrob� vektor bit� stejn� d�lky, jako je po�et z�znamu v prim�rn�m souboru, kde jedni�ka na $i$-t� pozici indikuje, �e $i$-t� z�znam m� pr�v� tuto hodnotu atributu. To umo��uje jednoduch� prov�d�n� booleovsk�ch dotaz� na tento atribut. Vektory bit� nav�c lze komprimovat, tak�e nezab�raj� tolik m�sta.
\end{obecne}

\begin{obecne}{Soubor s p��m�m p��stupem}
V tomto sch�matu jsou z�znamy v prim�rn�m souboru (\uv{adresov�m prostoru} velikosti $M$) rozpt�leny pomoc� \emph{hashovac� funkce}. �asto se pou��v� funkce $h=k\mod M'$, kde $M'$ je nejbli��� prvo��slo men�� ne� velikost adresov�ho prostoru. Hashovac� funkc� se ur�uje bu� jenom ��slo str�nky, nebo i relativn� pozice v n�. P�i hashov�n� vznikaj� kolize, kter� se daj� �e�it \emph{otev�en�m adresov�n�m} (�et�zen�m kolizn�ch z�znam� za sebe), \emph{rehashov�n�m} (dal�� funkc�) nebo pou�it�m \emph{oblasti p�ete�en�}. Snaha je v�t�inou um�stit kolizn� z�znamy do stejn� str�nky. 

Pokud je hashovac� funkce prost�, jedn� se o \emph{perfektn� hashov�n�}. Toho ale v praxi vlastn� nelze dos�hnout, tak�e se tento v�raz pou��v� i pro ozna�en� stavu, kdy je pro nalezen� z�znamu pot�eba nejv�� $O(1)$ p��stup� k m�diu. O�ek�van� d�lka �et�zce koliz� p�i po�tu $N$ z�znam� v prostoru velikosti $M$ je $1/(1-\frac{N}{M})$.
\end{obecne}

\subsubsection*{T��d�n� na vn�j�� pam�ti}

\begin{e}{Algoritmus}{0}{T��d�n� sl�v�n�m (Mergesort)}
Tento algoritmus se pou��v� pro t��d�n� dat, kter� se nevejdou do vnit�n� pam�ti. D� se pou��t i p�i sekven�n�m p��stupu k datov�m soubor�m. Nejjednodu��� verze bez buffer� vypad� takto:
\begin{pitemize}
    \item inicializace: na za��tku ka�d�ho kroku data rozd�l� do 2 soubor�
    \item na�te 2 z�znamy, ka�d� z jednoho souboru a porovn� je
    \item ve spr�vn�m po�ad� je zap�e do v�stupn�ho souboru, ze vstupn�ho souboru si na�te dal�� dva
    \item v prvn�m kroku z�sk�m uspo��dan� posloupnosti d�lky 2; v dal��ch kroc�ch v�dy porovn�m na�ten� prvky, zap�u men�� z nich a ze souboru odkud tento poch�zel si na�tu dal��, tak�e z�sk�m v�dy uspo��dan� posloupnosti dvojn�sobn� d�lky ne� v p�edchoz�m kroku
    \item po $\lceil\log n\rceil$ kroc�ch je soubor s $n$ z�znamy set��d�n�.
\end{pitemize}
Vylep�en� se dos�hne nap�. p��mo st��dav�m zapisov�n�m v�stupu do 2 soubor�, kdy se zbav�m nutnosti na za��tku ka�d�ho kroku data d�lit, nebo pou�it�m v�ce soubor� najednou. Je taky mo�n� vyu��t rostouc�ch posloupnost� prvk�, kter� se v souboru nach�zej� ji� p�ed zapo�et�m t��d�n�.
\end{e}

\begin{e}{Algoritmus}{0}{T��d�n� haldou}
Pro t��d�n� ve vnit�n� pam�ti se pou��v� algoritmus \emph{t��d�n� haldou (heapsort)}, kter� se d� zakomponovat do vylep�en� t��d�n� sl�v�n�m (viz n�e). Jeho z�kladem je datov� struktura \emph{halda} (konkr�tn� maxim�ln� halda, max-heap), reprezentovan� jako pole z�znam�, na kter�m je bin�rn� stromov� struktura: z�znam $k$ m� v�dy vy��� kl�� ne� jeho dva synov�, nach�zej�c� se na pozic�ch $2k+1$ a $2k+2$ p�i ��slov�n� od $0$ (pokud tato pozice nen� v�t�� ne� velikost haldy, v opa�n�m p��pad� z�znam nem� syny). Na pozici $0$ se tak nach�z� z�znam s nejvy���m kl��em. Postup t��d�n� je n�sledovn�:
\begin{pitemize}
    \item nejv�t�� prvek (z pozice $0$) se prohod� s t�m prvkem, jeho� ��slo pozice odpov�d� aktu�ln� velikosti haldy
    \item velikost haldy se zmen�� o 1
    \item dokud neplat� podm�nka, �e kl�� prvku z�skan�ho z konce haldy je v�t�� ne� oba kl��e jeho syn�, prohazuje se tento se synem s v�t��m kl��em (a tak posouv� v hald� d�l)
    \item toto se opakuje, dokud je velikost haldy v�t�� ne� 1, odzadu tak v poli vznik� set��d�n� posloupnost
\end{pitemize}
�asov� slo�itost algoritmu je $O(n\cdot\log n)$ pro pole z�znam� velikosti $n$.
\end{e}

\begin{e}{Algoritmus}{0}{$n$-cestn� t��d�n�}
Pokud m�m k dispozici ve vnit�n� pam�ti $n+1$ str�nek, mohu postupovat n�sledovn�:
\begin{pitemize}
    \item v 1. kroku na��st do pam�ti $n$ str�nek
    \item ty set��dit pomoc� heapsortu (nebo i quicksortu apod.) a z�skat tak del�� set��d�n� �seky (\emph{b�hy})
    \item sl�vat v�dy $n$ nejkrat��ch b�h� (pomoc� mergesortu) a vytv��et tak jeden b�h
    \item toto opakovat, dokud existuje v�ce ne� 1 b�h.
\end{pitemize}
�as. slo�itost pro $M$ str�nek v souboru je $O(2M\lceil\log_n M/n\rceil)$.
\end{e}

\begin{e}{Algoritmus}{0}{Dvojit� halda}
Del�� b�hy p�i sl�v�n� se daj� vytv��et pomoc� dvojit� haldy -- v pam�ti m�m dv� haldy z celkem $n$ prvk�, opakovan� z prvn� haldy odeb�r�m a zapisuji minim�ln� prvek do v�stupn�ho b�hu a na��t�m dal�� prvek, pokud ten je v�t�� ne� minimum haldy, vlo��m ho do prv� haldy, pokud je men��, vlo��m ho do druh� haldy, kter� vznik� od konce m�ho pole. A� se prvn� halda vy�erp�, pou�iji druhou a za�nu nov� b�h. Toto v nejhor��m p��pad� d�v� stejnou velikost b�h� jako oby�ejn� halda, pr�m�rn� je 2x lep��.
\end{e}


\subsubsection*{B-stromy}

\begin{e}{Definice}{0}{B-strom}
B-strom ��du $m$ je v��kov� vyv�en� strom, kter� m� n�sl. vlastnosti:
\begin{penumerate}
    \item Ko�en m� minim�ln� 2 syny, pokud nen� s�m listem.
    \item Ka�d� jin� uzel krom� list� m� nejm�n� $\lceil\frac{m}{2}\rceil$ a nejv�ce $m$ syn� a v�dy o 1 m�n� dat. z�znam� (listy maj� jen datov� z�znamy).
    \item Kl��e v�ech z�znam� v $i$-t�m podstromu uzlu $A$ jsou v�t�� ne� kl�� $i$-t�ho z�znamu uzlu $A$ a men�� nebo rovny kl��i $i+1$-t�ho z�znamu.
    \item v�echny \emph{v�tve} (cesty od ko�ene k listu) jsou stejn� dlouh�.
\end{penumerate}
Variantou jsou \emph{redundantn� B-stromy}, kdy v�echna data jsou um�st�na v listech, vnit�n� uzly obsahuj� pouze vyhled�vac� kl��e. Jin� mo�nost je pou�it� pouze kl��e a odkazu na cel� z�znam, m�sto vkl�d�n� kompletn�ch z�znam� do stromu.
\end{e}

\begin{e}{Algoritmus}{0}{Operace na B-strom�}
\emph{Vyhled�v�n�} v B-stromech podle kl��e se prov�d� jednoduch�m pr�chodem do hloubky.

\emph{Vkl�d�n�} prob�h� tak, �e se najde m�sto, kam z�znam vlo�it, pokud nen� uzel pln�, prost� se z�znam vlo��, jinak se uzel roz�t�p�, p�lka prvk� se d� vlevo, p�lka vpravo a prost�edn� se vlo�� (�mezi n�) do otce. Pokud v otci nen� m�sto, pokra�uje se stejn�m zp�sobem a� do ko�ene, kde se p��padn� vytvo�� nov� uzel a ud�l� se z n�j ko�en.

\emph{Odeb�r�n�} prvk� je opa�n� postup, v p��pad� podte�en� uzlu (z�stane v n�m m�n� ne� $\lceil\frac{m}{2}\rceil$ syn�) mus�m p�eb�rat data od sousedn�ch uzl� nebo sl�vat. V redundantn�ch B-stromech nen� nutn� p�i maz�n� odstra�ovat vyhled�vac� kl�� z vnit�n�ch uzl� -- prvek s touto hodnotou se ve strom� u� nebude nach�zet, ale vyhled�vat podle jeho kl��e je d�l mo�n�.
\par
Lep�� napln�nosti uzl� za cenu sn�en� rychlosti se d� dos�hnout pomoc� \emph{vyva�ov�n� str�nek} -- p�i p�ete�en� str�nky nejd��v kontroluji, jestli nejsou voln� sousedn�; pokud ano, p�erozd�l�m data a uprav�m kl��e. Podobn� je mo�n� postupovat p�i maz�n� (i pokud nen� t�eba sl�vat).
\par
Dal��m vylep�en�m je odlo�en� �t�pen� -- ke ka�d�mu listu nebo skupin� list� p��slu�� str�nka p�ete�en�, kam se vkl�daj� z�znamy, kter� se u� do dan�ho m�sta nevejdou. Nov� vkl�d�n� a �t�pen� je provedeno a� tehdy, jestli�e se str�nka p�ete�en� i v�echny p��slu�n� uzly napln�. Takto upraven� strom s v�ce ne� 1 �rovn� m� v�dy v�echny listy zapln�n� (za p�edpokladu nepou�it� maz�n�). P��slu��-li str�nky p�ete�en� skupin�m list�, mus�m je p�i maz�n� a p�id�v�n� list� takt� �t�pit a sl�vat.
\end{e}

\begin{e}{Definice}{0}{B+ stromy}
\emph{B+ stromy} jsou m�rn�m vylep�en�m B-strom� pro zrychlen� intervalov�ch dotaz�: v�echny uzly ve stejn� �rovni (a nebo jenom listy) jsou spojeny do spojov�ho seznamu (mo�n� je jednosm�rn� i obousm�rn� varianta).
\end{e}

\begin{e}{Definice}{0}{B* stromy}
\emph{B* stromy} (��du $m$) jsou �pravou B-strom� na z�klad� vyva�ov�n� str�nek. Druh� podm�nka B-strom� se uprav� tak, �e ka�d� uzel krom� ko�ene a list� m� minim�ln� $\lceil(2m-1)/3\rceil$ a max. $m$ syn� a odpov�daj�c� po�et dat. z�znam�. Listy maj� op�t jen stejn� rozmez� pro po�et dat. z�znam�. P�i vkl�d�n� prvk� se st�pen� odkl�d� op�t do t� doby, dokud nejsou pln� i sourozenci dan�ho listu; potom se �t�p� bu� 2 listy do 3, nebo 3 do 4 (bu� s pomoc� jednoho nebo dvou sousedn�ch sourozenc�). Odeb�r�n� podobn� zahrnuje sl�v�n� 3 uzl� do 2 (nebo 4 do 3). P�i ob�m lze ve slo�it�j�� variant� zapojit je�t� v�ce uzl�.
\end{e}

\begin{e}{Definice}{0}{Prefixov� stromy (Trie)}
Tento druh strom� slou�� k ulo�en� dat, kl��ovan�ch �et�zci. Jde o redundantn� stromy, data jsou ulo�ena a� v listech; vyhled�vac� kl��e jsou v�dy co nejkrat�� mo�n� prefixy �et�zc�, nutn� k odli�en� uzl�. Cel� hodnoty kl��� (a dal�� data) se nach�zej� a� v listech. P�i vkl�d�n� a �t�pen� str�nek se n�jakou heuristikou hled� nejkrat�� prefix, kter� by vznikl� str�nky odd�lil. Vylep�en� varianta neukl�d� u syn� p�edponu kl��e, kterou m� rodi� -- je to pam�ov� efektivn�j��, ale zvy�uje v�po�etn� n�roky.
\end{e}

\begin{e}{Definice}{0}{Stromy s prom�nnou d�lkou z�znamu}
Jde o modifikaci B-stromu, kter� umo��uje do n�j ulo�it z�znamy prom�nn� d�lky. Listy se ne�t�p� podle po�tu z�znam�, ale zhruba na poloviny podle velikosti dat. Druh� podm�nka B-strom� se uprav� n�sledovn�: celkov� d�lka z�znam� v jednom uzlu je minim�ln� $\lceil B/2\rceil$ a maxim�ln� $B$ (kde $B$ je n�jak� zvolen� hodnota, v�t�. velikost str�nky na disku). Existuje i varianta s podm�nkou \uv{$2/3$}, jako maj� B*-stromy.

Probl�mem t�to struktury je tendence del��ch z�znam� ke stoup�n� ke ko�eni, ��m� se sni�uje arita z�znam�. To se �e�� hled�n�m d�l�c�ho z�znamu s min. d�lkou tak, aby vznikl� uzly spl�ovaly podm�nky stromu (a je to docela n�ro�n�). Nav�c �t�pen� je slo�it�j�� -- 1 str�nka se m��e �t�pit na 3 (vlo��m-li hodn� dlouh� z�znam), m��e doj�t ke zmen�en� stromu p�i vlo�en� apod., b�n� se pou��v� obecn� algoritmus nahrazov�n�, jeho� speci�ln� p��pady jsou insert a delete.
\end{e}


\begin{e}{Definice}{0}{V�cerozm�rn� B-stromy}
Pou��vaj� se, je-li pot�eba efektivn� hledat z�znamy podle v�ce atribut�. Jde o propojenou mno�inu strom�. K jednotliv�m atribut�m p��slu�ej� prvky pole odkaz� na seznamy strom�, ve kter�ch se podle dan�ch atribut� d� hledat. Pro prvn� atribut je pot�eba jen 1 strom, v n�m je pro ka�d� kl�� odkaz na cel� strom 2. atributu (pro dal�� je to podobn�). Stromy stejn�ho atributu jsou ve spojov�m seznamu. Mohu hledat v�echny z�znamy, pro kter� zn�m hodnoty v�ech atribut�, nebo jenom jejich podmno�inu -- vy�aduje to proj�t v�ce strom�, ale nen� t�eba mno�inov�ch operac�.
\end{e}


\end{document}
